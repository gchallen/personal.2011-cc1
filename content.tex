% <wc:start description="Safety v. Liability" max=800>

Since my undergraduate years a decade ago Harvard has struggled to balance
the pleasures and dangers of alcohol on campus. Many of the changes I've
witnessed are definite improvements. For me, AlcoholEdu was the evening I
spent kneeling in front of the toilet, and my DAPA was my freshman girlfriend
cradling my head above it. I arrived at college clueless about alcohol and
armed with little more than trial and error to figure it out. A trip to the
Ad Board (trial) and many mixed drinks (error) later, I started to get the
hang of things. Today we have an Office of Alcohol and Other Drug Services
which supports many good things: AlcoholEdu, the DAPA program, Ryan Travia
and something called e-TOKE which I have not experimented with but have been
assured is not a joke.

Yet, despite these new programs, alcohol on campus remains dangerous, as
evidenced by the sobering statistics recently released about alcohol-related
hospital admissions last semester. DAPA claims to avoid scare tactics, but
Provost Hyman and Dean Hammonds did their best to scare the assembled
Resident Tutors with these numbers at the unusual mid-year tutor meeting.
Perhaps they frightened themselves: Provost and Dean both left before the
event concluded with a panel discussion of the issue they had raised. After
an Adams House tutor delivered a moving declaration of his love for the party
form, the panel came to a general agreement that alcohol remains worrisome.

% 11 Feb 2011 : GWA : From Seth:
%               By taking an unrealistic and paternalistic approach to
%               drinking on campus, the administration has driven students to
%               drink in more irresponsible ways. When I was an
%               undergraduate, we had many more opportunities to drink a few
%               beers at local bars or Stein Clubs whereas now students feel
%               they have to pound shots in their rooms. Every other
%               university in the nation I've ever been to takes a much more
%               reasoned and less confrontational approach to drinking,
%               whereas Harvard seems more concerned about being sued than
%               ensuring the growth and welfare of its students. Maybe these
%               other schools all just have better lawyers, but I suspect the
%               issue is that we, as campus leaders and administrators, need
%               to find the courage to pursue polices that acknowledge
%               reality, encourage students to have a good time while
%               drinking responsibly, and don't simply aim to insulate
%               ourselves from lawsuits and helicopter parents.

Before the meeting concluded, however, Quincy House tutor Seth Moulton '01
raised an uncomfortable question highlighting the tensions inherent to
Harvard's approach to campus drinking. He pointed out that underage students
from our era had opportunities to drink at supervised, house-sponsored events
like Stein Club where our imbibing was tolerated in the hopes that we might
learn how to drink responsibly.

In my case, the approach worked: I did learn to drink. I remember
encountering a blockmate in the basement of Lowell House. ``There's alcohol
in the Grille,'' he said. ``It's very strange.'' We were both underage, and
there was alcohol available in the grille. But it wasn't strange. It was the
comfortable public alcohol that brings friends together, not the furtive drug
taken to shot-pounding extremes behind closed doors. Hoisting a pint in
responsible company formed the cornerstone of my alcohol education.

Today Harvard doesn't tolerate those grey areas, and BAT teams patrol Stein
Club. We focus on student safety as long as it doesn't incur any liability.
When it does, we turn away from those we are trying to educate, pursuing
confrontational and ineffective strategies. Two examples illustrate my point:
the legal drinking age and the Harvard-Yale Tailgate.

Why won't Harvard support lowering the legal drinking age? We hear repeatedly
that 21 is Massachusetts state law, and that is true. But ``Don't Ask, Don't
Tell'' was Federal law and Harvard made it's disagreement with that misguided
policy extremely clear and extremely public. The mixture of legal and
underage drinkers on campus is the source of many of our difficulties with
alcohol. Students hide their parties because inevitably some of the attendees
are underage. Tutors stay away because confronting underage drinking strands
us between the reality of campus drinking and our own liabilities.

Trying to defend the current drinking age is hopeless, but Harvard won't
officially decriminalize underage alcohol use on campus. It won't publicly
eliminate underage drinking as an disciplinary priority. It won't indemnify
residential life staff against alcohol-related lawsuits, freeing us to focus
on student safety. Harvard won't even join the 135 other institutions as a
signatory to the Amethyst Initiative, which merely calls for ``debate on the
21 year-old drinking age.'' In the face of this silence, it's easy to reach
the uncomfortable conclusion that Harvard won't act because it cares about
liability more than safety. Because Harvard worries that a lowered drinking
age will mean more drinking, more accidents, and more lawsuits.

If Harvard cared about student safety it would increase the length of the
Harvard-Yale tailgate. Every year the tailgate gets shorter, more
restrictive, and less safe. It's just not hard to posit a connection between
the tailgate length and student safety. The shorter the tailgate, the faster
the drinking; the faster the drinking, the less safe the event. Why won't
Harvard sponsor a more relaxed, more normal tailgate, like the one they have
at Yale? Perhaps because a shorter tailgate also limits the amount of time
that Harvard is sponsoring an alcohol-driven event, reducing their liability.

What's needed is something else Seth's question mentioned: institutional
courage. Courage to reconcile safety v. liability concerns in favor of safety
and students, and take a few unpopular countercultural stances along the way.
Courage the College had ten years ago, but seems to have lost.

% <wc:end>

\textit{Geoffrey Challen '02--'03 is a Resident Tutor at Eliot House. Views
expressed are his and do not reflect official Harvard College policy.}
