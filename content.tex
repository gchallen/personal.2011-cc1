Since my days as an undergraduate Harvard has confronted the dangers of
alcohol on campus. Many of these developments are definite improvements from
the way things used to be. For me, AlcoholEdu was the evening I spent
crawling back and forth to the toilet, and my DAPA was my freshman girlfriend
holding my head above it. I hadn't a clue about alcohol in high school and
arrived at college armed with little more than trial and error to figure
liquor out. A good helping of error and a trip to the Ad Board later, I
started to get the hang of things. Today we have an Office of Alcohol and
Other Drug Services which does a number of good things: providing AlcoholEdu,
running the DAPA program, employing Ryan Travia and offering something called
e-TOKE which I have not experimented with but have been assured is not a
joke.

And yet despite these advances, alcohol on campus remains dangerous, as
evidenced by the sobering statistics recently released about alcohol-related
hospital admissions this last fall. The College's refusal to release these
statistics along with information that might help put the numbers into
context is frustrating, and despite DAPA claiming to avoid scare tactics,
they administration did seem to feel the statistics were useful to scare the
Resident Tutors. At an unusual mid-year residential life staff meeting,
Provost Hyman and Dean Hammonds presented these statistics to Tutors from
across the college. Then they promptly left. The event closed with a panel of
veteran tutors agreeing that this was a problem and we should think about it.

% 11 Feb 2011 : GWA : From Seth:
%               By taking an unrealistic and paternalistic approach to
%               drinking on campus, the administration has driven students to
%               drink in more irresponsible ways. When I was an
%               undergraduate, we had many more opportunities to drink a few
%               beers at local bars or Stein Clubs whereas now students feel
%               they have to pound shots in their rooms. Every other
%               university in the nation I've ever been to takes a much more
%               reasoned and less confrontational approach to drinking,
%               whereas Harvard seems more concerned about being sued than
%               ensuring the growth and welfare of its students. Maybe these
%               other schools all just have better lawyers, but I suspect the
%               issue is that we, as campus leaders and administrators, need
%               to find the courage to pursue polices that acknowledge
%               reality, encourage students to have a good time while
%               drinking responsibly, and don't simply aim to insulate
%               ourselves from lawsuits and helicopter parents.

Before the meeting concluded, however, Quincy House tutor Seth Moulton '01
raised a question highlighting some of the tensions that play out regarding
Harvard's approach to liquor on campus. He pointed out that a decade ago
Harvard provided undergraduates opportunities to drink---both legally and
illegally---at supervised, house-sponsored events like Stein Club where
underage drinking used to be tolerated in the hopes that students might learn
how to drink responsibly.

And at least, in my case, I did learn how to drink. I remember distinctly the
night I encountered a blockmate in the basement of Lowell House, near the
mailboxes. ``There's alcohol in the Grille,'' he said. ``It's very strange.''
He was underage, I was underage, there was alcohol for us in the grille, but
it wasn't strange. Alcohol is a healthy part of a social experience that
brings friends together, not a furtive thing taken to shot-pounding extremes
behind closed doors. And in fact, hoisting a pint or two under the watchful
eyes of house tutors became a very normal experience for me, and today I am a
better drinker because of the (illegal) lessons I learned at Lowell Stein
Club.

Today Harvard doesn't tolerate those grey areas---BAT teams patrol Stein
Club. We are good at focusing on student safety as long as it doesn't incur
any University liability. When it does, we turn away from those we are trying
to educate and claim to care about, pursuing confrontational and ineffective
strategies. A few examples will illustrate my point: the legal drinking age,
the clubs, and the Harvard Tailgate.

If Harvard cared about student safety it would lead or join campaigns to
return the drinking age to 18. The mixture of legal and underage drinkers on
campus is the source of most of our difficulties with alcohol. Students don't
want tutors patrolling their parties in fear that some or most of the
attendees are underage. And tutors don't drop by parties or Stein Club
because confronting underage drinking maroons us between the reality of
campus drinking and our own liabilities. Lowering the drinking age would also
allow all students to together patronize the bars within walking distance,
where trained bartenders monitor their clientele as they meter out drinks.

Why won't Harvard support lowering the legal drinking age? The line we hear
repeatly is that 21 is Massachusetts state law, and that is true. But ``Don't
Ask, Don't Tell'' was Federal law and Harvard made it's disagreement with
that misguided policy extremely clear and extremely public. Trying to defend
the current drinking age is hopeless, but Harvard won't publicly
decriminalize underage drinking on campus. It won't completely eliminate
underage drinking as an institutional disciplinary priority. It won't promise
to protect house masters, tutors, and other residential life staff in case of
lawsuits emanating from alcohol-related accidents on campus, freeing us to
focus on student safety. Harvard won't even join the 135 other institutions
as a signatory to the Amethyst Initiative, which only goes as far as
encouraging ``informed and unimpeded debate on the 21 year-old drinking
age.'' In the face of this silence, it's hard to come to any other conclusion
that Harvard won't act because it cares about liability more than safety.
Because Harvard worries that a lowered drinking age will mean more drinking,
more accidents, and more lawsuits.

If Harvard cared about student safety it would aggressively patrol the Final
Clubs surrounding campus. After all, this is where a large amount of
unmonitored and underage student drinking takes place. Why won't it? Because
the clubs aren't Harvard property, and improving campus safely might annoy
club alumni and their deep pockets.

If Harvard cared about student safety it would increase the length of the
Harvard-Yale tailgate. Every year the tailgate gets less safe, and every year
the University response is to reduce the length or restrict the kinds of
alcohol served. Harvard's persistence in pursuing a failed strategy would be
more impressive if it were less difficult to posit a connection between a
longer tailgate and student safety. The shorter the tailgate, the faster the
drinking; the faster the drinking, the less safe the event. Why won't Harvard
sponsor a more relaxed, more normal Harvard-Yale tailgate? Because a shorter
tailgate also limits the amount of time that Harvard is ``officially''
sponsoring an alcohol-driven event, reducing their liability.

At the end of the day it comes down to another key virtue that Seth
mentioned: courage. Courage to reconcile safety v. liability concerns in
favor of safety and students, and take a few unpopular countercultural
stances along the way.
